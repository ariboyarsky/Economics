% --------------------------------------------------------------
% This is all preamble stuff that you don't have to worry about.
% Head down to where it says "Start here"
% --------------------------------------------------------------
 
\documentclass[12pt]{article}
 
\usepackage[utf8]{inputenc}
\usepackage[margin=1in]{geometry} 
\usepackage{amsmath,amsthm,amssymb,amsfonts,enumerate}

\newcommand{\N}{\mathbb{N}}
\newcommand{\Z}{\mathbb{Z}}
\newcommand{\Q}{\mathbb{Q}}
\newcommand{\R}{\mathbb{R}}

\newcommand\inner[2]{\langle #1, #2 \rangle}
\newcommand\norm[1]{\| #1 \|}

\let\oldforall\forall
\let\forall\undefined
\DeclareMathOperator{\forall}{\,\oldforall\,}

\let\oldexists\exists
\let\exists\undefined
\DeclareMathOperator{\exists}{\,\oldexists\,}
 
\newenvironment{theorem}[2][Theorem]{\begin{trivlist}
\item[\hskip \labelsep {\bfseries #1}\hskip \labelsep {\bfseries #2.}]}{\end{trivlist}}
\newenvironment{lemma}[2][Lemma]{\begin{trivlist}
\item[\hskip \labelsep {\bfseries #1}\hskip \labelsep {\bfseries #2.}]}{\end{trivlist}}
\newenvironment{exercise}[2][Exercise]{\begin{trivlist}
\item[\hskip \labelsep {\bfseries #1}\hskip \labelsep {\bfseries #2.}]}{\end{trivlist}}
\newenvironment{reflection}[2][Reflection]{\begin{trivlist}
\item[\hskip \labelsep {\bfseries #1}\hskip \labelsep {\bfseries #2.}]}{\end{trivlist}}
\newenvironment{proposition}[2][Proposition]{\begin{trivlist}
\item[\hskip \labelsep {\bfseries #1}\hskip \labelsep {\bfseries #2.}]}{\end{trivlist}}
\newenvironment{corollary}[2][Corollary]{\begin{trivlist}
\item[\hskip \labelsep {\bfseries #1}\hskip \labelsep {\bfseries #2.}]}{\end{trivlist}}
 

\begin{document}

\title{Notes from MWG's \emph{Microeconomic Theory}} %replace X with the appropriate number
\author{Ariel Boyarsky\\ aboyarsky@uchciago.edu} %if necessary, replace with your course title
 
\maketitle

% --------------------------------------------------------------
%                         Start here
% --------------------------------------------------------------
\section*{Ch. 1 Preference and Choice}
\subsection{Preference Relations}
Axioms of Preference:
\begin{enumerate}[i.]
	\item $x \succ y$. x is "strictly" preffered to y \emph{if and only if (iff)} $x \succeq y$ but $y \not\succeq x$. 
	\emph{Note:} $x \succeq y$ is an atleast as good as relation. That is, x is at least as good as y. 
	\item $x \sim y$. Then,  x is indifferent to y. Thus,  $x \succeq y$ and $y \succeq x$.
\end{enumerate}

\textbf{Definition:} We call a relation \emph{rational} if it is both \textbf{complete} and \textbf{transitive}.

Consider the set of goods, $X$.
\begin{enumerate}[i.]
	\item Completeness: $\forall x,y \in X$ either $x \succeq y$ or $y \succeq x$ or both.
	\item Transitivity: $\forall x,y,z \in X$ if $x \succeq y$ and $y \succeq z$ then $x \succeq z$.
\end{enumerate}

\textbf{Proposition:} It follows, that if $\succeq$ is rational then:
\begin{enumerate}[i.]
\item $\succ$ is irreflexive
\end{enumerate}
\section*{Ch. 2 Consumer Choice}
Walrasian budget set, $X = \R_+^L$



% ---------------------------------------------------------------
\end{document}
% ---------------------------------------------------------------