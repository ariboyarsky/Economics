% --------------------------------------------------------------
% Ariel Boyarsky 
% aboyarsky@uchicago.edu
% --------------------------------------------------------------
 
\documentclass[12pt]{article}
 
\usepackage[utf8]{inputenc}
\usepackage[margin=1in]{geometry} 
\usepackage{amsmath,amsthm,amssymb,amsfonts,enumerate}
\usepackage[hidelinks]{hyperref}


\newcommand{\N}{\mathbb{N}}
\newcommand{\Z}{\mathbb{Z}}
\newcommand{\Q}{\mathbb{Q}}
\newcommand{\R}{\mathbb{R}}

\newcommand\inner[2]{\langle #1, #2 \rangle}
\newcommand\norm[1]{\| #1 \|}

\let\oldforall\forall
\let\forall\undefined

\let\bf\oldbf
\let\bf\textbf
\DeclareMathOperator{\forall}{\,\oldforall\,}

\let\oldexists\exists
\let\exists\undefined
\DeclareMathOperator{\exists}{\,\oldexists\,}
 
\newenvironment{theorem}[2][Theorem]{\begin{trivlist}
\item[\hskip \labelsep {\bfseries #1}\hskip \labelsep {\bfseries #2.}]}{\end{trivlist}}
\newenvironment{lemma}[2][Lemma]{\begin{trivlist}
\item[\hskip \labelsep {\bfseries #1}\hskip \labelsep {\bfseries #2.}]}{\end{trivlist}}
\newenvironment{exercise}[2][Exercise]{\begin{trivlist}
\item[\hskip \labelsep {\bfseries #1}\hskip \labelsep {\bfseries #2.}]}{\end{trivlist}}
\newenvironment{reflection}[2][Reflection]{\begin{trivlist}
\item[\hskip \labelsep {\bfseries #1}\hskip \labelsep {\bfseries #2.}]}{\end{trivlist}}
\newenvironment{proposition}[2][Proposition]{\begin{trivlist}
\item[\hskip \labelsep {\bfseries #1}\hskip \labelsep {\bfseries #2.}]}{\end{trivlist}}
\newenvironment{corollary}[2][Corollary]{\begin{trivlist}
\item[\hskip \labelsep {\bfseries #1}\hskip \labelsep {\bfseries #2.}]}{\end{trivlist}}

\makeatletter
\renewcommand\thesection{}
\renewcommand\thesubsection{\@arabic\c@section.\@arabic\c@subsection}
\makeatother
 
% Change Abstract Name
\renewcommand{\abstractname}{Brief Introduction}


\begin{document}

\title{Notes from MWG's \emph{Microeconomic Theory \\ Draft 1}} 
\author{Ariel Boyarsky\\ aboyarsky@uchciago.edu} 
\maketitle

% --------------------------------------------------------------
%                         Begin Article
% --------------------------------------------------------------

\begin{abstract}
The following are select notes from \emph{Micoreconomic Theory} by Mas-Colell, Whinston, and Green. I have tried to synthesize these into the key axioms, theorems, propositions, and proofs introduced in the text. Overall we mirror the Definition-Theorem-Proof style employed by MWG. Examples and notes are provided where needed. Some proofs are my own work and thus prone to error. Additionally, I have attempted to explicit show steps and maintain strict notation, if you spot any issues with notation or other ambiguities, please let me know! Please remember this is an individual effort and very much a work in progress. It should not be viewed as a replacement for the original text. Additionally, in some areas we draw from the Microeconomic Theory notes of Noah Miller to add further context (\href{https://business.illinois.edu/nmiller/documents/notes/firsthalf.pdf}{\emph{available here}}).

Please send comments or concerns to aboyarsky [at] uchicago [dot] edu.
\end{abstract}
\section{Ch. 1 Preference and Choice}
\subsection{Preference Relations}
Axioms of Preference:
\begin{enumerate}[i.]
	\item $x \succ y$. x is "strictly" preferred to y \emph{if and only if (iff)} $x \succeq y$ but $y \not\succeq x$. 
	\emph{Note:} $x \succeq y$ is an at least as good as relation. That is, x is at least as good as y. 
	\item $x \sim y$. Then,  x is indifferent to y. Thus,  $x \succeq y$ and $y \succeq x$.
\end{enumerate}

\textbf{Definition:} We call a relation \emph{rational} if it is both \textbf{complete} and \textbf{transitive}.

Consider the set of goods, $X$.
\begin{enumerate}[i.]
	\item Completeness: $\forall x,y \in X$ either $x \succeq y$ or $y \succeq x$ or both.
	\item Transitivity: $\forall x,y,z \in X$ if $x \succeq y$ and $y \succeq z$ then $x \succeq z$.
\end{enumerate}

\textbf{Proposition 1.B.1:} It follows that if $\succeq$ is rational then:
\begin{enumerate}[i.]
\item $\succ$ is irreflexive and transitive
\item $\sim$ is reflexive and transitive
\item If $x \succ y \succeq z$ then $x \succ z$
\end{enumerate}


\begin{proof} \textbf{Proposition 1.B.1}
\\Consider the set of goods X. Then take $x,y,z \in X$. 
\\Let $\succeq$ be rational so it is both complete and transitive.
\begin{enumerate}[i.]
\item Suppose by contradiction that $x \succ x$. Then, since $x \succeq x$ and $x \not\succeq x$. Both these facts cannot be true by the def of a strict pref relation. relation, so $\succ$ is irreflexive. Furthermore, consider $x \succ y$ and $y \succ x$. Then, $ x \succeq y$ and $y \not\succeq x$. Also, $ y \succeq z$ and $z \not\succeq y$. So, since $\succeq$ is rational, by transitivity we have $x \succ z$.
\item Consider $x \sim x$. Then,  $x \succeq x$ and $x \succeq x$. If condition one holds than condition two must hold. So, $\sim$ is reflexive. Also, if $x \sim y$ and $y \sim z$ then, $x \succeq y$ and $y \succeq x$, and $z \succeq y$ and $y \succeq z$. So, by transitivity $x \sim z$.
\item If $x \succ y $ then $x \succeq y$ and $y \not\succeq x$. Also, $y \succeq z$. Then, by transitivity $x \succeq z$. But, since $y \not\succeq x$ then by transitivity $z \not\succeq x$. So, $x \succ z$.
\end{enumerate}
\end{proof}
\noindent
If a relation is rational we can also represent it with a \textbf{utility function}:
\\\textbf{\\Definition} $u: X \rightarrow \R$ is a utility function if $\forall x,y \in X$, $x\succeq y \; $ iff $ \; u(x) \geq u(y)$.
Notice, that this map is not unique.

\textbf{\\Proposition:} A preference relation must be rational to represent it with a utility function.

\emph{\\Proof Sketch.} Since, $u(\cdot)$ is real-valued $\forall x,y \in X \; u(x) \geq u(y)$ or $u(y) \geq u(x)$. This implies completeness. Then, $\forall x,y,z \in X$ if $u(x) \geq u(y)$ and $u(y) \geq u(z)$ then, $u(x) \geq u(z)$. This implies transitivity. 

\subsection{Choice}
\textbf{Definitions}
A choice structure is defined by $(\mathcal{B}, C(\cdot))$.
\begin{enumerate}
\item $\mathcal{B}$ is a family on nonempty subsets of X. That is, $B \in \mathcal{B}, B \subset X$. Intuitively, $\mathcal{B}$ is a family of comprehensive budget sets that that are possible in a society. However, it is not necessarily all possible subsets in X.
\item $C(\cdot)$ is a choice rule. It assigns a nonempty set of chosen elements, $C(B) \subset B$. These are the decision makers chosen alternatives given a budget set.
\end{enumerate}
\textbf{Example:}
$X = \{x,y,z\}, \mathcal{B} = \{\{x,y\}, \{x,z\}\}$ 
\\Then one possible $C(\{x,y\}) = \{x\}$ and $C(\{x,z\}) = \{x, z\}$. 
\\We can apply the weak axiom of revealed preference (Samuelson 1947) so that we may expect some consistency. We can formally write this as:
\\\textbf{Definition of Weak Axiom of Revealed Preference:} Given $(\mathcal{B}, C(\cdot))$. The choice structure satisfies the weak axiom of revealed preference if $x,y \in B$ and $x \in C(B)$ then $\forall B' \in \mathcal{B} \; s.t. \; x,y \in B'$. If $y \in C(B')$ then, $x\in C(B')$.
\\ Perhaps, a simpler yet equivalent statement is that x is revealed at least as good as y if and only if there is some budget set, B, where x is in C(B).
\\ More formally, the \underline{weak axiom} states $x \succeq\star \; y \; \text{iff} \; B \in \mathcal{B} \; s.t. \; x,y \in B$ and $x \in C(B)$.
\subsection{Interplay of Preference and Choice}

Consider $C^*(B, \succeq) = \{x \in B | x \succeq y \forall y \in B\}$. This represents the most preferred good in the budget set (could be null in theory). We assume it is nonempty.
\\Thus, we generate the choice structure $(\mathcal{B}, C^*(B, \succeq))$.
\\ \textbf{\\Proposition:} $(\mathcal{B}, C^*(B, \succeq))$ satisfies the weak axiom of revealed preference if $\succeq$ is a rational preference relation.
\begin{proof}
Take $x \in C^*(B, \succeq)$ for some $x,y \in B \in \mathcal{B}$. Then, $x \succeq y$. Now consider by contradiction $x,y \in B'$ and $y \in C^*(B', \succeq)$. Then, $y \succeq z \forall z \in B'$. Then, this implies that $y \succeq x$ which contradicts our initial construction. Thus, if $y \in C^*(B', \succeq)$ then also $x \in C^*(B', \succeq)$. Thus, this relation satisfies the weak axiom. 
\end{proof}
\textbf{\\Definition:} Given a choice structure we say a rational preference relation \underline{rationalizes} $C(\cdot)$ relative to $\mathcal{B}$ if 
$$C(B) = C^*(B, \succeq) \forall B \in \mathcal{B}$$. 
\\That is, the rational preference relation $\succeq$ rationalizes the choice rule on $\mathcal{B}$ if it generates the same output as the preference maximizing choice rule ($C^*(\cdot, \succeq)$). We then call the choice maker a preference maximizer.
\\\textbf{\\Example:} Notice that the following choice structure satisfies the weak axiom but is not rationalizing preferences: $\mathcal{B} = \{\{x,y\}, \{y,z\}, \{z,x\}\}, \; C(\{x,y\}) = x, \; C(\{y,z\}) = y, \; C(\{z,x\}) = z$.
\\ \textbf{\\Proposition:} If $(\mathcal{B}, C(\cdot))$ is a choice structure such that:
\begin{enumerate}[1.]
	\item the weak axiom is satisfied,
	\item $\mathcal{B}$ includes all subsets of X up to three elements,
\end{enumerate}
then there exists a \underline{unique} rational preference relation $\succeq$ that rationalizes $C(\cdot)$ such that 
$$C(B) = C^*(B, \succeq) \forall B\in \mathcal{B}$$
\\ \emph{\\Proof Sketch:} To prove the above result we must (1) show $\succeq*$ is rational, (2) $\succeq*$ rationalizes $C(\cdot)$, and (3) the solution is unique (this falls from every binary pairing being included in $\mathcal{B}$).

\section{Ch. 2 Consumer Choice}
The consumer is the most basic decision unit in the economy. We assume a market economy where commodities (good and services) are traded and available for purchase.

\subsection{Commodities}
We assume a finite number of commodities, $L$.
\\ The commodity vector lists amounts of each commodity:
\\ $$x = \begin{bmatrix} 
x_1 \\
\vdots \\
x_L 
\end{bmatrix}$$

\subsection{Consumption Set}
The consumption set is an element of the commodity space, $X \subset \R^L$. 
\\The consumption set represents an individuals conceivable consumption given particular constraints.
\\ The consumption set is defined as $X = \R_+^L = \{x \in \R^L | x_l \geq 0 \forall 1,\dots,L\}$
\\ It is a set of all nonnegative bundles of commodities.
\\ Notice that $\R^L$ is convex. Take $x,x' \in \R^L$ then $x'' = \alpha x + (1-\alpha)x' \in \R^L \forall \alpha \in [0,1]$.
\\ This design allows us to reflect physical constraints on the consumption bundle.

\subsection{Budgets}
We also add a monetary constraint to the consumption set.
\\ We assume completeness or universality of markets and that consumers are price takers. 
\\ Then we define the price vector as an element in $\R^L$
$$p = \begin{bmatrix} 
p_1 \\
\vdots \\
p_L 
\end{bmatrix} \in \R^L$$
\\ For simplicity we assume non-negativity of prices.
\\ Finally, we introduce a consumers total wealth, $w$. That is,
$$\bf{p}\cdot \bf{x} = p_1x_1 + \dots + p_Lx_L \leq w\footnote{Notice we use the dot product here.}$$
\\The consumer problem is simply to choose a consumption bundle $x \in B_{\bf{p}, w}$.
\\ \bf{\\Definition:} Since there are multiple consumption bundles that exhibit this property for a given price vector and wealth we define the \emph{Walrasian} budget set or \emph{competitive budget set} as:
$$B_{\bf{p},w} = \{\bf{x} \in \R^L_+ \;|\; \bf{p}\cdot \bf{x} \leq w\}$$
\\ \bf{\\Definition:} We define the budget hyperplane (or line in $L=2$) to be the set of consumption bundles defined by $\{\bf{x} \in \R^L_+ \;|\; \bf{p}\cdot \bf{x} = w\}$.
\vspace{10pt}
\\There are multiple interesting phenomena associated with the budget hyperplane. First, in $\R^2$ the slope of the line is $m = -\frac{(w/p_2)}{(w/p_1)} = -(p_1/p_2)$. Notice that the negativity of the slope reflects the budget constraint.
\vspace{10pt}
\\Furthermore, the price vector starting at any $x \in \{\bf{x} \in \R^L_+ \;|\; \bf{p}\cdot \bf{x} = w\}$ is orthogonal to any vector on the budget line. This is because $p\cdot x = p\cdot x' = w$. That is,
$$p\cdot x - p \cdot x' = 0 $$
$$ p\cdot(x - x') = 0$$
$$p \cdot \Delta x = 0$$
Furthermore, $B_{\bf{p}, w}$ is a convex set. Consider $\bf{p} \cdot \bf{x}'' = \bf{p}\cdot \alpha \bf{x} + \bf{p} \cdot (1-\alpha)\bf{x}' \leq w$.\footnote{This is easily proven if we recall $\forall \bf{x} \in B_{\bf{p}, w}, \; \bf{p}\cdot \bf{x} \leq w$} 
\subsection{Comparative Statics}
\bf{Definition:} $x(\bf{p}, w)$ defines a Walrasian demand correspondence (also called a \bf{choice function}\footnote{In many ways this is interchangeable with the notion of an ordinary or Marshallian demand function that we study later.}). That is, given a price vector and consumer wealth, this function gives us the consumers choice of commodity bundle. Thus,

$$x(\bf{p}, w): \R^{L+1}\rightarrow\R^L$$
Notice, that $L + 1$ refers to number of prices (of goods, $L$) and the wealth argument, $w$.
\vspace{10pt}
\\ We assume the choice function is \emph{homogeneous of degree zero} and satisfies \emph{Walras' Law}.
\\ \bf{\\Definition:} $x(\bf{p}, w)$ is homogeneous of degree zero if $x(\alpha\bf{p}, \alpha w) = x(\bf{p}, w) \forall \bf{p}, w, \alpha > 0$.
\\ That is, if price and wealth change in the same proportion there is no change since demand is based on feasibility. This entails that:
$$B_{\alpha\bf{p}, \alpha w} = B_{\bf{p}, w}$$
\bf{Definition:} $x(\bf{p}, w)$ satisfies Walras' Law if $\forall p >> 0 $ and $w > 0$:
$$\bf{p}\cdot\bf{x} = w \forall \bf{x} \in x(\bf{p}, w)$$
Simply, this says that the consumer fully expends their wealth over a lifetime. 
\\ Also notice, that this defines a choice structure $(\mathcal{B}^w, x(\cdot))$ given $\mathcal{B}^w = \{B_{\bf{p}, w} | \bf{p} >> 0, w > 0 \}$.
We can write the demand function in matrix form with an entry for each good as such:
$$x(\bf{p}, w) = \begin{bmatrix} 
x_1(\bf{p}, w) \\
\vdots \\
x_L(\bf{p}, w) 
\end{bmatrix} \in \R^L$$
Furthermore, it is useful to note that $\bf{p}\cdot x(\bf{p}, w) = w$ is more than an equality, it is an identity:\footnote{A key feature of an identity is that the equivalence still holds if we differentiate both sides. Consider 2a=1, if we differentiate both sides with respect to $a$ we get 2=0. But, if we convert this to $2a=z$ such that $a(z) =\frac{z}{2}$ and substitute such that $2a(z) \equiv z$ (now no matter what value of z we choose the relationship holds) and so, now we can differentiate both sides to get $2a'(z) = 1$ or $a'(z) = 1/2$. Thus, if we increase z by 1 we increase a(z) by 1/2.}
$$\bf{p}\cdot x(p,w) \equiv w$$
$$\sum^L_{l=1}p_lx_l(\bf{p}, w) \equiv w$$
Thus, we can differentiate both sides:
$$\frac{\partial}{\partial w}\sum^L_{l=1}p_lx_l(\bf{p}, w) = \frac{\partial}{\partial w}w$$
$$\sum^L_{l=1}p_l\frac{\partial x_l(\bf{p}, w)}{\partial w} = \frac{d}{dw}w$$
$$\sum^L_{l=1}p_l\frac{\partial x_l(\bf{p}, w)}{\partial w} = 1$$
That is, if we increase wealth by 1, we also increase consumption by 1 which is a statement of Walras' Law.
\vspace{10pt}
\\Comparative statics study choices under changes in various economic parameters such as wealth and price.
\\\bf{\\Wealth Effects}
\\ First, some definitions:
\\ \bf{Definition:} If we fix $\bf{p}^*$ then $x(\bf{p}^*, w)$ is called the \emph{Engel function}.
\\ \bf{\\Definition:} The image of the Engel function, $E_{\bf{p}^*} = \{x(\bf{p}^*,w) \; | \; w > 0\}$, is called the \emph{wealth expansion function}.
\\ \bf{\\Definition:} $\frac{\partial x_l(\bf{p},w)}{\partial w}$ is the \emph{wealth effect} for the $l$'th good.
\\ \bf{\\Definition:} If $\frac{\partial x_l(\bf{p},w)}{\partial w} > 0$ then the good is \emph{normal} otherwise if $< 0$ it is \emph{inferior} at $(\bf{p}, w)$. Furthermore, if every good is normal we say demand is normal.
\\ We can express the wealth effects at any $(\bf{p}, w)$ for goods using matrix notation:\footnote{We use the $D_xf(x,y)$ notation to denote the matrix of partial derivatives of $f$ with respect to x.}
$$D_wx(\bf{p}, w) = \begin{bmatrix} 
\frac{\partial x_1(\bf{p},w)}{\partial w} \\
\vdots \\
\frac{\partial x_L(\bf{p},w)}{\partial w} 
\end{bmatrix} \in R^L$$
\\ \bf{Price Effects}
\\ \bf{Definition:} $\partial x_l(\bf{p}, w)/\partial p_k$ is the price effect of the price of good k on demand for good l. As the price of good $k$ changes how does the price of good $l$?
\\ \bf{\\Definition:} If $\partial x_l(\bf{p}, w)/\partial p_l > 0$ then we call it a \emph{giffen good}. That is as price rises, so too does consumption.
\\ Notice, that we can conveniently represent price effects in matrix form:
$$D_p x(\bf{p}, w) = \begin{bmatrix} 
\frac{\partial x_1(\bf{p}, w)}{\partial p_1} & \dots & \frac{\partial x_1(\bf{p}, w)}{\partial p_L}\\
 \vdots & \ddots & \vdots \\
\frac{\partial x_L(\bf{p}, w)}{\partial p_1} & \dots & \frac{\partial x_L(\bf{p}, w)}{\partial p_L} 
\end{bmatrix}$$
\bf{Proposition:} If the Walrasian demand function is homogeneous of degree 0 then for all p and w: 
$$\sum^L_{j=1} p_j \frac{\partial x(\bf{p}, w)}{\partial p_j} + w\frac{\partial x(\bf{p}, w)}{\partial w} = 0$$
$$D_px(\bf{p}, w)p + D_wx(\bf{p}, w)w = 0$$

\begin{proof}
If the demand function is homogeneous of degree 0 then,
$$x_l(\alpha\bf{p}, \alpha w) \equiv x_l(\bf{p}, w) \forall l, \alpha > 0$$
Now we set $\bf{p} = \bf{p}*$, $w = w*$ and take the \emph{total} derivative with respect to $\alpha$,
$$x_l(\alpha\bf{p}*, \alpha w*) \equiv x_l(\bf{p}*, w*)$$
$$\frac{\partial x_l(\alpha \bf{p}*, \alpha w*)}{\partial \bf{p}}\cdot\frac{\partial \bf{p}*}{\partial \alpha} + \frac{\partial x_l(\alpha \bf{p}*, \alpha w*)}{\partial w}\frac{\partial w*}{\partial \alpha} = 0$$
Now, consider that $\alpha \bf{p}* = \bf{p}$ so, $\frac{\partial \bf{p}}{\partial \alpha} = p*$, similarly $\frac{\partial w}{\partial \alpha} = w*$. Substituting we get,
$$\frac{\partial x_l(\alpha \bf{p}*, \alpha w*)}{\partial \bf{p}}\cdot \bf{p}*+ \frac{\partial x_l(\alpha p*, \alpha w*)}{\partial w}w* = 0$$
Now, set $\alpha = 1$ and since $p*$ is a vector of prices we can rewrite this in the form presented by \emph{MWG}:\footnote{Notice, MWG drops the star notation.}
\begin{equation}
\sum^L_{j=1}\bf{p}_j^*\frac{\partial x_l( \bf{p}*, w*)}{\partial \bf{p}} + \frac{\partial x_l( \bf{p}*,  w*)}{\partial w}w* = 0
\end{equation}

\end{proof}
From this equation we can all define elasticity of consumer demand:
\\\bf{\\Definition:} We define elasticity of demand with respect to price and wealth respectively:
$$\epsilon_{lp_j} = \frac{\%\Delta x_i}{\%\Delta p_j} = \frac{\partial x_l(\bf{p}, w)}{\partial p_j}\frac{p_j}{x_l(\bf{p}, w)}$$
and
$$\epsilon_{lw} = \frac{\partial x_l(\bf{p}, w)}{\partial w}\frac{w}{x_l(\bf{p}, w)}$$
That is, the percent change in demand for good $l$ per percent change in price of good $j$. And percent change in demand for good $l$ per percent change in wealth. It is often to think of these as ($(\Delta x/x)/(\Delta w/w)$).
\\ Returning to equation (1) we can divide by $x_l{\bf{p}, w}$ to get this equation in terms of elasticities. Resulting in:
\begin{equation}\epsilon_{lw} + \sum^L_{j = 1}\epsilon_{lp_j} = 0\end{equation}
This says that if we sum the \% change in consumer demand due to a \% percent change in wealth and prices across each good is 0. That is, there is no change in demand if prices and wealth change in proportion. This is given by homogeneity of degree zero. To quote MWG, "an equal percentage change in all prices and wealth leads to no change in demand."
\\ \bf{\\Proposition:} If the Walrasian demand function satisfies Walras' law then,
\begin{equation}
\sum^L_{l=1}p_l\frac{\partial x_l(p,w)}{\partial p_k} + x_k(\bf{p}, w) = 0 \forall k = 1,\dots,L
\end{equation}
This is sometimes called the \emph{Cournot aggregation}. It states that if the price of good $k$ increases we must spend more of our wealth on it. Thus, we rearrange our demand for the other goods such that our total expenditure does not change. We prove the statement below:
\begin{proof}
Recall, $\bf{p}\cdot x(\bf{p}, w) \equiv w$ then,
$$\sum^L_{l=1}p_lx_l(\bf{p},w) \equiv w$$
$$\frac{\partial}{\partial p_k}\sum^L_{l=1}p_lx_l(\bf{p},w) \equiv \frac{\partial}{\partial p_k}w$$
Consider that at some $l = k$ we can rewrite the above equation as:
$$\frac{\partial}{\partial p_k}\sum^L_{l=1, l\not=k}p_lx_l(\bf{p},w) + \frac{\partial}{\partial p_k}p_kx_k(\bf{p},w)\equiv \frac{\partial}{\partial p_k}w$$
Differentiating gives,
$$x_k(\bf{p}, w) + p_k\frac{\partial x_k(\bf{p}, w)}{p_k} + \sum^L_{l=1, l\not= k}p_l\frac{\partial x_l(\bf{p},w)}{\partial p_k} \equiv 0$$
We can combine terms to get,
$$x_k(\bf{p}, w) + \sum^L_{l=1}p_l\frac{\partial x_l(\bf{p},w)}{\partial p_k} \equiv 0$$
\end{proof}
Similarly differentiating with respect to wealth provides us with a further implication of Walras' law:
\\ \bf{Proposition:} If the Walrasian demand function satisfies Walras' law then: 
\begin{equation}\sum^L_{l=1}p_l\frac{\partial x_l(\bf{p}, w)}{\partial w} \equiv 1\end{equation}
This is often called the \emph{Engel aggregation}. This simply states that if we increase consumer wealth by 1, expenditure will also increase by 1. That is, expenditure and wealth must change by equal amounts.
\begin{proof}
Recall, $\bf{p}\cdot x(\bf{p}, w) \equiv w$ then,
$$\sum^L_{l=1}p_lx_l(\bf{p},w) \equiv w$$
$$\frac{\partial}{\partial w}\sum^L_{l=1}p_lx_l(\bf{p},w) \equiv \frac{\partial}{\partial w}w$$
$$\sum^L_{l=1}p_l\frac{\partial x_l(\bf{p}, w)}{\partial w} \equiv 1$$
\end{proof}
% ---------------------------------------------------------------
\end{document}
% ---------------------------------------------------------------