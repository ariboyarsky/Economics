\documentclass[twoside]{article}
\setlength{\oddsidemargin}{0.25 in}
\setlength{\evensidemargin}{0.25 in}
\setlength{\topmargin}{-0.6 in}
\setlength{\textwidth}{6.5 in}
\setlength{\textheight}{8.5 in}
\setlength{\headsep}{0.75 in}
\setlength{\parindent}{0 in}
\setlength{\parskip}{0.1 in}

%
% ADD PACKAGES here:
%


\usepackage{amsfonts,graphicx}
\usepackage{mathtools}
\usepackage[linewidth=0.5pt]{mdframed}


%
% The following commands set up the lecnum (lecture number)
% counter and make various numbering schemes work relative
% to the lecture number.
%
% \newcounter{section}
\renewcommand{\thepage}{\arabic{page}}
\renewcommand{\thesection}{\arabic{section}}
\renewcommand{\theequation}{\arabic{equation}}
\renewcommand{\thefigure}{\arabic{figure}}
\renewcommand{\thetable}{\arabic{table}}

%
% The following macro is used to generate the header.
%
\newcommand{\lecture}[1]{
   %\pagestyle{myheadings}
   %\thispagestyle{plain}
   \newpage
   % \setcounter{lecnum}{#1}
   % \setcounter{page}{1}
   \noindent
   \begin{center}
   \framebox{
      \vbox{\vspace{2mm}
    \hbox to 6.28in { {
	\hfill Last Updated: \textit{\today} } }
       \vspace{4mm}
       \hbox to 6.28in { {\Large \hfill Overview of Useful Econometric Tools  \hfill} }
       \vspace{2mm}
       \hbox to 6.28in { {\it By: #1 \hfill} }
      \vspace{2mm}}
   }
   \end{center}
   %\markboth{Lecture #1: #2}{Lecture #1: #2}

 
   \vspace*{4mm}
}

\renewcommand{\cite}[1]{[#1]}
\def\beginrefs{\begin{list}%
        {[\arabic{equation}]}{\usecounter{equation}
         \setlength{\leftmargin}{2.0truecm}\setlength{\labelsep}{0.4truecm}%
         \setlength{\labelwidth}{1.6truecm}}}
\def\endrefs{\end{list}}
\def\bibentry#1{\item[\hbox{[#1]}]}

%Use this command for a figure; it puts a figure in wherever you want it.
%usage: \fig{NUMBER}{SPACE-IN-INCHES}{CAPTION}
\newcommand{\fig}[3]{
			\vspace{#2}
			\begin{center}
			Figure \thelecnum.#1:~#3
			\end{center}
	}


\let\span\undefined
\newcommand\span[1]{\text{span}(#1)}
\newcommand\rank[1]{\text{rank}(#1)}
\newcommand\im[1]{\text{im}(#1)}
\newcommand\diag[1]{\text{diag}(#1)}
\newcommand\tr[1]{\text{tr}#1}
\newcommand\argmax[1]{\text{argmax}#1}


\usepackage{etoolbox}
\newcommand{\define}[4]{\expandafter#1\csname#3#4\endcsname{#2{#4}}}
\forcsvlist{\define{\newcommand}{\mathbb}{}}{N,Z,Q,R,C,F,G,T,A,B,D,E}

% Use these for theorems, lemmas, proofs, etc.

\newtheorem{theorem}{Theorem}[section]
\newtheorem{lemma}[theorem]{Lemma}
\newtheorem{proposition}[theorem]{Proposition}
\newtheorem{claim}[theorem]{Claim}
\newtheorem{corollary}[theorem]{Corollary}
\newtheorem{definition}[theorem]{Definition}
\newenvironment{proof}{{\bf Proof:}}{\hfill\rule{2mm}{2mm}}

% **** IF YOU WANT TO DEFINE ADDITIONAL MACROS FOR YOURSELF, PUT THEM HERE:

\begin{document}
%\lecture{**AUTHOR**}
\lecture{Ariel Boyarsky (aboyarsky@uchicago.edu)}

\textbf{Preface}

The following are a diverse set of econometric tools useful in economic analysis. Each section gives and overview of a different tool. The main purpose is to serve as a quick reference. Each section is written to be independent of the others. I will also try to add useful papers and book chapters to serve as more complete references. Please note this is a work and progress and I will add to it as time goes on. Please feel free to email me with any mistakes.


\tableofcontents

\clearpage

\section{Hedonic Regression}

This technique is common in evaluating real estate assets.

Suppose the price of a product are defined by,

\begin{equation}\label{eq:hedon_model}
p_i  = f(z_{i,1}, \dots, z_{i,k}, e)
\end{equation}

\textbf{Assumptions:} $\E[e_{i}|z_{i}] = 0$ (Mean Independence - needed for unbiasedness), Homogeneity of Preferences, Information of Features, No Multico (A possible problem if there is a large number of categorical variables).

Where the $z$'s define $k$ characteristics and e is a random error term of unobservables.

In housing price models it is often common to assume these characteristics may be sorted into structural $s$, location $n$, and environmental $v$ characteristics. As such one may often see the model,

$$P = f(s,n,v)$$

Our goal is to get the marginal contributions of each characteristic to price. As such, we need to make a parametric assumption to identify these contributions. In housing price analysis this is often a linearity assumption as in Equation \ref{eq:lin_assump}. In other goods we can also specify a log-linear model as shown in Equation \ref{eq:log_assump} - this is often done for high tech codes. This form is useful if we believe there is a multiplicative relationship between the features.

\begin{equation}\label{eq:lin_assump}
p_i  = \beta_0 + \sum^K_{k=1}\beta_kz_{ik} + u
\end{equation}
\begin{equation}\label{eq:log_assump}
\ln p_i  = \beta_0 + \sum^K_{k=1}\beta_kz_{ik} + u
\end{equation}

In the second stage we estimate the willingness of the consumer to pay. As such,

$$p_i = g(X, Y, u)$$

Where $X$ are the covariates of the product, $Y$ is the consumer income, and $u$ are individual level unobservables which we often are assume proxy for tastes.


\subsection{References}

Diewert (2003). Hedonic Regression: A Consumer Theory Approach.



% **** END OF FILE **** %
\end{document}





